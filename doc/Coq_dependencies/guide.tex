\documentclass[11pt, a4paper,UKenglish,cleveref, autoref, thm-restate]{article}
\usepackage{babel}
\usepackage{varioref}

\usepackage{geometry}
\geometry{noheadfoot, vmargin=1.0in, hmargin=0.5in}
\usepackage[ddmmyyyy]{datetime}
\usepackage{prerex}

\usetikzlibrary{fit}

\title{\bf A Guide for the Structure of the Coq Libraries for Zooid}
\author{David Castro (Imperial College London, University of Kent) \and Francisco Ferreira (Imperial College London) \and Lorenzo Gheri (Imperial College London)\and Nobuko Yoshida (Imperial College London)}
\begin{document}

\maketitle

%\thispagestyle{empty}


Here we give a brief overview on the structure of the Coq libraries, in the artifact for the paper ``Zooid: a DSL for Certified Multiparty Computation''; these can be found in the folder \texttt{theories}. Figure \ref{fig:dep} shows the dependency graphs of the libraries; we follow the graph on the left, which gives a structure of the main development. More detail can be found in the paper and its appendix, with pointers to the code for definitions and theorems.\\

The file \texttt{Common.v} calls the theories in the folder \texttt{Common}.
\begin{itemize}
\item \texttt{Forall.v} and \texttt{Member.v} provide support for lists, which will be used for continuations (branching) in inductively defined global and local types.
\item In \texttt{AtomSets.v} \emph{payload types} for sent/received messages in Zooid are defined.
\item \texttt{Alt.v} provides support for partial functions, as we use these for continuations (branching) in coinductively defined global and local trees.
\item In \texttt{Actions.v} the \emph{actions} of our LTS semantics are defined, together with the codatatype of \emph{execution traces}.
\item \emph{Queue environments} for asynchronous semantics are defined in \texttt{Queue.v}.
\end{itemize}

The file \texttt{Global.v} calls the theories in the folder \texttt{Global}.
\begin{itemize}
\item In \texttt{Syntax.v} the dadatype of \emph{global types} is defined; some basic infrastructure is given.
\item In \texttt{Tree.v} the codadatype of \emph{global trees} is defined; some basic infrastructure is given.
\item \texttt{Unravel.v} gives the \emph{unravelling relation}, that pairs global types with their representation as trees.
\item In \texttt{Semantics.v} we give the semantics for global types: we define a \emph{labelled transition system} for global trees (unravelling of global types) and we introduce the notion of \emph{execution trace}.
\end{itemize}

The file \texttt{Local.v} calls the theories in the folder \texttt{Local}.
\begin{itemize}
\item In \texttt{Syntax.v} the dadatype of \emph{local types} is defined; some basic infrastructure is given.
\item In \texttt{Tree.v} the codadatype of \emph{local trees} is defined; some basic infrastructure is given.
\item \texttt{Unravel.v} gives the \emph{unravelling relation}, that pairs local types with their representation as trees.
\item In \texttt{Semantics.v} we give the semantics for local types: we define a \emph{labelled transition system} for local trees (unravelling of local types) and we introduce the notion of \emph{execution trace}.
\end{itemize}

The file \texttt{Projection.v} calls the theories in the folder \texttt{Projection}.
\begin{itemize}
\item In \texttt{IProject.v} the inductive \emph{projection of a global type onto a participant}, is defined as a recursive function, outputting a local type.
\item In \texttt{CProject.v} the coinductive \emph{projection of a global tree onto a participant}, is defined as a relation, between a global tree and a local tree.
\item \texttt{Correctness.v} contains the theory, leading to theorem \texttt{ic\_proj}, namely ``unravelling preserves projection''.
\item In \texttt{QProject.v} Defines the \emph{projection of a global tree on queue environments}, which is fundamental for the following results, connecting the asynchronous LTS semantics of global and local trees.
\end{itemize}

The file \texttt{TraceEquiv.v} concludes the metatheory of multiparty session types. It contains \emph{soundness and completeness} results(\texttt{Project\_step} and \texttt{project\_lstep} respectively) and finally \emph{trace equivalence} for global trees and local environments (\texttt{TraceEquivalence}).\\

In \texttt{Proc.v} the \emph{syntax and semantics of Zooid processes} are defined, together with the \emph{typing system}, as a relation between processes and local types. The file contains the two central theorems on which Zooid's DSL is built: \emph{type preservation} (\texttt{preservation}) and \emph{protocol compliance}, in terms of obedience of execution traces of processes obeying the discipline of a global type  (\texttt{process\_traces\_are\_global\_types}). Also, in the file \texttt{Proc.v}, the infrastructure for code extraction is given.\\

In \texttt{Zooid.v} we set up our DSL: \emph{Zooid terms} are dependent pairs of a process and a proof that it is well-typed with respect to a given local type. The library provides smart constructors, helper functions and notations, in order to enable building these well-typed-by-construction terms. \\

Finally, the three final files: \texttt{Examples.v}, \texttt{Tutorial.v}, and \texttt{Code.v} display, through examples, the applicability of Zooid. As examples, we implement recursive pipeline, simple calculator service, and the two buyer-protocol, a common benchmark of multiparty session types. As tutorial, we implement recursive ping-pong, the server and several clients that illustrate how to program using Zooid. While \texttt{Code.v} performs the code extraction to get OCaml implementations of all the examples and tutorials and connects it to the OCaml runtime.


\setcounter{diagheight}{90}


\begin{figure}
\begin{chart}%\grid

%main
\reqhalfcourse 15,75:{}{Common.v}{}
\reqhalfcourse 5,65:{}{Global.v}{}
\reqhalfcourse 25,65:{}{Local.v}{}
\reqhalfcourse 15,55:{}{Projection.v}{}
\reqhalfcourse 15,45:{}{TraceEquiv.v}{}
\reqhalfcourse 15,35:{}{Proc.v}{}
\reqhalfcourse 15,25:{}{Zooid.v}{}
\reqhalfcourse 15,15:{}{Examples.v}{}
%Common.v
\reqhalfcourse 45,85:{}{Common/Forall.v}{}
\reqhalfcourse 45,77:{}{Common/Member.v}{}
\reqhalfcourse 48,69:{}{Common/Alt.v}{}
\reqhalfcourse 65,77:{}{Common/AtomSets.v}{}
\reqhalfcourse 65,69:{}{Common/Actions.v}{}
\reqhalfcourse 82,69:{}{Common/Queue.v}{}
%Global.v
\reqhalfcourse 47,60:{}{Global/Syntax.v}{}
\reqhalfcourse 67,60:{}{Global/Trees.v}{}
\reqhalfcourse 57,52:{}{Global/Unravel.v}{}
\reqhalfcourse 77,50:{}{Global/Semantics.v}{}
%Local.v
\reqhalfcourse 47,41:{}{Local/Syntax.v}{}
\reqhalfcourse 67,41:{}{Local/Trees.v}{}
\reqhalfcourse 57,33:{}{Local/Unravel.v}{}
\reqhalfcourse 77,31:{}{Local/Semantics.v}{}
%Projection.v
\reqhalfcourse 43,22:{}{Projection/IProject.v}{}
\reqhalfcourse 63,22:{}{Projection/CProject.v}{}
\reqhalfcourse 78,14:{}{Projection/QProject.v}{}
\reqhalfcourse 53,5:{}{Projection/Correctness.v}{}
%tricks
\mini 36,88:{}
\mini 36,66:{}
\mini 40,63:{}
\mini 40,47:{}
\mini 40,44:{}
\mini 40,28:{}
\mini 34,25:{}
\mini 34,02:{}

%main
\prereq 15,75,25,65:
\prereq 15,75,5,65:
\prereq 5,65,15,55:
\prereq 25,65,15,55:
\prereq 15,55,15,45:
\prereq 15,45,15,35:
\prereq 15,35,15,25:
\prereq 15,25,15,15:
%Common.v
\prereq 45,85,45,77:
\prereq 45,77,48,69:
\prereq 65,77,48,69:
\prereq 65,77,65,69:
\prereq 65,77,82,69:
\coreq 15,75,36,88:
\coreq 15,75,36,66:
%Global.v
\prereq 47,60,57,52:
\prereq 67,60,57,52:
\prereq 67,60,77,50:
\coreq 5,65,40,63:
\coreq 5,65,40,47:
%Local.v
\prereq 47,41,57,33:
\prereq 67,41,57,33:
\prereq 67,41,77,31:
\coreq 25,65,40,44:
\coreq 25,65,40,28:
%Projection.v
\prereq 43,22,53,5:
\prereq 63,22,53,5:
\coreq 15,55,34,25:
\coreq 15,55,34,02:

%Common.v
\begin{pgfonlayer}{courses}
\draw[dashed] ([shift={(-3mm,1mm)}]x45y85.north west) rectangle ([shift={(1mm,-1mm)}]x82y69.south east);
\end{pgfonlayer}
%Global.v
\begin{pgfonlayer}{courses}
\draw[dashed] ([shift={(-1mm,1mm)}]x47y60.north west) rectangle ([shift={(1mm,-1mm)}]x77y50.south east);
\end{pgfonlayer}
%Local.v
\begin{pgfonlayer}{courses}
\draw[dashed] ([shift={(-1mm,1mm)}]x47y41.north west) rectangle ([shift={(1mm,-1mm)}]x77y31.south east);
\end{pgfonlayer}
%Projection.v
\begin{pgfonlayer}{courses}
\draw[dashed] ([shift={(-1mm,1mm)}]x43y22.north west) rectangle ([shift={(50mm,-1mm)}]x53y5.south east);
\end{pgfonlayer}

\end{chart}
\caption{Dependency Graph for Zooid's Coq Libraries}
\label{fig:dep}
\end{figure}


\end{document}
